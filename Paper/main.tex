\documentclass{article}
\usepackage[utf8]{inputenc}
\usepackage{fullpage,amsmath,amssymb}
\usepackage{graphicx,caption, float}
\usepackage{fancyvrb,placeins,lscape, arydshln}
\usepackage[title]{appendix}
\usepackage{newtxtext,newtxmath, url}
\RecustomVerbatimCommand{\VerbatimInput}{VerbatimInput}{fontsize=\scriptsize}
\newenvironment{changemargin}[2]{%

\begin{list}{}{%
\setlength{\topsep}{0pt}%
\setlength{\leftmargin}{#1}%
\setlength{\rightmargin}{#2}%
\setlength{\listparindent}{\parindent}%
\setlength{\itemindent}{\parindent}%
\setlength{\parsep}{\parskip}%
}%
\item[]}{\end{list}}

\title{\textit{Get Pumped:} Water Table Data}
\author{Ethan Buck, Leyli Garryyeva, Gergana Gospodinova, and Xin Zhang}
\date{\today}

\begin{document}

\maketitle

\section{Problem Description}
This project stems from finding a data set from a competition on DrivenData.  The competition is called ``Pump it Up: Data Mining the Water Table'', and provides a data set with different water points.  There are six different types of water points:

\begin{enumerate}
    \item Cattle Trough
    \item Communal Standpipe
    \item Dam
    \item Hand Pump
    \item Improved Spring
    \item Other
\end{enumerate}

The goal is to accurately predict the operating condition of these water points located in Tanzania.  There are three different classes for this response variable, which are

\begin{enumerate}
    \item functional
    \item functional needs repair
    \item non functional
\end{enumerate}

Thus, this is a classification problem.  There are 39 different predictors for this data set, nine of which are continuous covariates.  Early data exploration reveals that none of these continuous covariates appear to resemble a normal distribution, and thus LDA and QDA most likely are not suitable choices.  However, an adapted logistic regression may prove to be fruitful in this analysis, as well as a potential random forest model.  KNN classification may be suitable as well, since there are a large number of training observations (59,400).

Section \ref{dataClean} will discuss the data cleaning process.  Exploratory plots are shown in Section \ref{explore}.  The KNN analysis follows in Section \ref{knn}.  Section \ref{logistic} goes through the logistic regression results, and Section \ref{randForest} explains the results from random forests.

\section{Data Cleaning} \label{dataClean}
Test test
$$7x^2 - 4$$

\section{Exploratory Analysis} \label{explore}
We wanted to first see if LDA and QDA were potential models by checking for normal covariates.  We saw that there were a large number of factors in the data, and so we tried only considering the numerical predictors, and used histograms to see if we could rule out normal distributions.  These are shown in Figure \ref{fig:NonNormalHists}.  It is clear that all of these are far from being able to be assumed to follow normal distributions, and thus we presume that LDA and QDA models are not appropriate and would not yield predictive models.

\begin{figure}[h]
    \centering
    \includegraphics[width=0.6\textwidth]{Figures/HistsNotNormal.png}
    \caption{Histograms of some of the numerical predictors.}
    \label{fig:NonNormalHists}
\end{figure}

\section{KNN Classification} \label{knn}


\section{Logistic Regression} \label{logistic}



\section{Random Forest Model} \label{randForest}


\section{Conclusions}

\newpage
\section{Appendix}


%The data set can be found online at \\ %\url{https://www.drivendata.org/competitions/7/pump-it-up-data-mining-the-water-table/page/23/}

\end{document}
