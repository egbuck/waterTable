\documentclass{article}

\usepackage[utf8]{inputenc}
\usepackage{fullpage,amsmath,amssymb,parskip}
\usepackage{graphicx,caption, float}
\usepackage{fancyvrb,placeins,lscape, arydshln}
\usepackage[title]{appendix}
\usepackage{newtxtext,newtxmath, url}
\RecustomVerbatimCommand{\VerbatimInput}{VerbatimInput}{fontsize=\scriptsize}

\title{\textit{Get Pumped:} Water Points in Tanzania}
\author{Ethan Buck, Leyli Garryyeva, Gergana Gospodinova, and Xin Zhang}
\date{\today}

\begin{document}

\maketitle

\section{Problem Description}
This project stems from finding a data set from a competition on DrivenData. The competition is called ``Pump it Up: Data Mining the Water Table'', and provides a data set with different water points. There are six different types of water points: cattle trough, communal standpipe, dam, hand pump, improved spring, and other.

The goal is to accurately predict the operating condition of these water points located in Tanzania. There are three different classes for this response variable, which are

\begin{enumerate}
    \item functional
    \item functional needs repair
    \item non functional
\end{enumerate}

Thus, this is a classification problem. There are 39 different predictors for this data set, nine of which are continuous covariates. Early data exploration reveals that none of these continuous covariates appear to resemble a normal distribution, and thus LDA and QDA most likely are not suitable choices. However, an adapted logistic regression may prove to be fruitful in this analysis, as well as a potential random forest model. KNN classification may be suitable as well, since there are a large number of training observations (59,400).

Section \ref{dataClean} will discuss the data cleaning process. Exploratory plots are shown in Section \ref{explore}. The KNN analysis follows in Section \ref{knn}. Section \ref{logistic} goes through the logistic regression results, and Section \ref{randForest} explains the results from random forests. In Section \ref{comparasion}, we are going to compare performance between models.

\section{Data Cleaning} \label{dataClean}
Test test
$$7x^2 - 4$$

\section{Exploratory Analysis} \label{explore}
We wanted to first see if LDA and QDA were potential models by checking for normal covariates.  We saw that there were a large number of factors in the data, and so we tried only considering the numerical predictors, and used histograms to see if we could rule out normal distributions.  These are shown in Figure \ref{fig:NonNormalHists}.  It is clear that all of these are far from being able to be assumed to follow normal distributions, and thus we presume that LDA and QDA models are not appropriate and would not yield predictive models.

\begin{figure}[h]
    \centering
    \includegraphics[width=0.6\textwidth]{Figures/HistsNotNormal.png}
    \caption{Histograms of some of the numerical predictors.}
    \label{fig:NonNormalHists}
\end{figure}

\section{KNN Classification} \label{knn}


\section{Logistic Regression} \label{logistic}



\section{Random Forest Model} \label{randForest}


\section{Comparing the Models} \label{comparasion}
Figure \ref{fig:modelComp} shows a bar graph comparing the performances of the different models.  The models seem to perform very similarly on the functional water points, as there are so many observations of this type.  The models appeared to suffer much more when trying to predict the ``functional needs repair'' class, as there are not many observations in this  category (comparatively).  However, the oversampling method with the random forest model did help to combat this. This method also comes close to doing the best on the non functional class, and these two classes are the most important to predict correctly, as the purpose is to identify broken water points.

\begin{figure}[!h]
    \centering
    \includegraphics[width = 0.7\textwidth]{Figures/percentAccuracyByClass.png}
    \caption{Comparing the models' accuracy of predicting each target class}
    \label{fig:modelComp}
\end{figure}

Overall, the random forest model outperformed the other fitting methods, and this was without requiring excessive tuning of the model.  The pre-processing of the data was enough to be able to fit a relatively accurate model, and with the addition of oversampling, we were able to achieve more balanced class accuracy rates.

\newpage
\section{Appendix}



%The data set can be found online at \\ %\url{https://www.drivendata.org/competitions/7/pump-it-up-data-mining-the-water-table/page/23/}

\end{document}